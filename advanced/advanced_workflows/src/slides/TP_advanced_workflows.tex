% =============================================================================
% File:  sample_slides.tex --  Example of the use of the Falkor Beamer theme
% Author(s): Sebastien Varrette <Sebastien.Varrette@uni.lu>
% Time-stamp: <Mar 2014-04-29 16:06 svarrette>
%
% Copyright (c) 2012 Sebastien Varrette <Sebastien.Varrette@uni.lu>
% .             http://varrette.gforge.uni.lu
%
% For more information:
% - LaTeX: http://www.latex-project.org/
% - Beamer: https://bitbucket.org/rivanvx/beamer/
% - LaTeX symbol list:
% http://www.ctan.org/tex-archive/info/symbols/comprehensive/symbols-a4.pdf
%
% Latest version of these files can be found on Github:
%

% =============================================================================
\documentclass{beamer}
% \documentclass[draft]{beamer}
\usepackage{_style}
\usepackage{__config}

% The key part to use my theme
\usetheme{Falkor}

% Not integrated in my theme as not everybody wants that
\AtBeginSection[]
{
  \frame{
    \frametitle{Summary}%
    {\scriptsize\tableofcontents[currentsection]}
  }
}

%%%%%%%%%%%%%%%%%%%%%%%%%%%% Header %%%%%%%%%%%%%%%%%%%%%%%%%%%%%%
\title{\EventName}
\subtitle{\TPindex: \TPtitle}

\author{\authors}
\institute[UL]{
  University of Luxembourg, Luxembourg
}

% Mandatory to define a logo - otherwise compilation will fail in an unobvious
% manner.
\pgfdeclareimage[height=0.8cm]{logo}{images/logo_UL.pdf}
%%\logo{\pgfuseimage{logo}}
\date{}

%%%%%%%%%%%%%%%%%%%%%%%%%%%%%% Body %%%%%%%%%%%%%%%%%%%%%%%%%%%%%%%
\begin{document}

\begin{frame}
    \vspace{2.5em}
    \titlepage
\end{frame}

% .......
\frame{
  \begin{center}
      \textbf{Latest versions available on
        \href{http://ulhpc-tutorials.readthedocs.org/latest/advanced/vm5k/README/}{ulhpc-tutorials.readthedoc.org}}:
      \vfill
      \begin{description}
        \item[UL HPC tutorials:] \hfill
          \myurl{https://github.com/ULHPC/tutorials}
        \item[UL HPC School:] \hfill
          \myurl{http://hpc.uni.lu/hpc-school/}
        \item[\TPindex tutorial sources:] \hfill
          \myurl{\TPghurl}
      \end{description}
  \end{center}
}

% ......
\frame{
  \frametitle{Summary}
  {\scriptsize
    \tableofcontents
  }
}

% ===============================================

\frame{
  \frametitle{Introduction}
  
  	Objectives
    \begin{itemize}
                \item Discover the advanced features of OAR
                \item Show how it can improve your workflows
                \item Use the advanced launcher scripts
        \end{itemize}

  \begin{exampleblock}{Follow the tutorial on readthedocs:}
      \begin{itemize}
        \item http://bit.ly/1Gwq3BL
      \end{itemize}
  \end{exampleblock}
}

% ===============================================
\section{Exercise 1: Advanced OAR features: container, array jobs}

\frame{
  \frametitle{Container jobs}

    \begin{itemize}
    			\item special job type
                \item a container is a pool of resources
                \item you can submit subjobs in a container
    \end{itemize}

	\cmdlinefrontend{oarsub -t container -l nodes=2 "sleep 1800"}
	
	\cmdlinefrontend{oarsub -I -t inner=<container id>}

}

\frame{
  \frametitle{Array jobs}
	
    \begin{itemize}
    			\item submit N jobs in one oarsub command
                \item split your workload according to the index
                \item other possibility: "--array-param-file"
    \end{itemize}

	\cmdlinefrontend{oarsub --array <N> -l /core=1 /path/to/prog.sh}
	
}

\section{Exercise 2: Best effort jobs}

\frame{
  \frametitle{Best effort}
    \begin{itemize}
                \item low priority
                \item overcome the limits (50 jobs in the default queue, 1000 in the besteffort queue)
                \item can be killed if the resources are required by the default queue
                \item killed jobs can be resubmitted automatically (idempotent)
                \item use a short walltime and resubmit the jobs until completion
    \end{itemize}

	\cmdlinefrontend{oarsub -t besteffort /path/to/prog}
	\cmdlinefrontend{oarsub -t besteffort -t idempotent /path/to/prog}

}

\section{Exercise 3: Checkpoint restart}

\frame{
  \frametitle{Checkpointing}
    \begin{itemize}
                \item save the state of an application
                \item be able to restart it from the saved state
                \item overcome the limits (walltime)
                \item bonus: fault tolerance
                \item case by base custom implementation, or generic with BLCR
    \end{itemize}

	\cmdlinefrontend{oarsub --checkpoint 30 --signal 12 -l walltime=00:02:00 -t besteffort -t idempotent /path/to/prog}
}

\section*{Thank you for your attention...}
\frame{
  \frametitle{Questions?}
  \begin{center}
      \includegraphics[scale=0.2]{question.jpg}
  \end{center}

  {\tiny
    \tableofcontents

  }
}

\end{document}

% ~~~~~~~~~~~~~~~~~~~~~~~~~~~~~~~~~~~~~~~~~~~~~~~~~~~~~~~~~~~~~~~~
% eof
%
% Local Variables:
% mode: latex
% mode: flyspell
% mode: auto-fill
% fill-column: 80
% End:
