% =============================================================================
% File:  sample_slides.tex --  Example of the use of the Falkor Beamer theme
% Author(s): Sebastien Varrette <Sebastien.Varrette@uni.lu>
% Time-stamp: <Mar 2014-04-29 16:06 svarrette>
% 
% Copyright (c) 2012 Sebastien Varrette <Sebastien.Varrette@uni.lu>
% .             http://varrette.gforge.uni.lu
% 
% For more information:
% - LaTeX: http://www.latex-project.org/
% - Beamer: https://bitbucket.org/rivanvx/beamer/
% - LaTeX symbol list:
% http://www.ctan.org/tex-archive/info/symbols/comprehensive/symbols-a4.pdf
% 
% Latest version of these files can be found on Github:
% 

% =============================================================================
\documentclass{beamer}
% \documentclass[draft]{beamer}
\usepackage{_style}
\usepackage{__config}

% The key part to use my theme
\usetheme{Falkor}

% Not integrated in my theme as not everybody wants that
\AtBeginSection[]
{
  \frame{
    \frametitle{Summary}
    {\scriptsize\tableofcontents[currentsection]}
  }
}

%%%%%%%%%%%%%%%%%%%%%%%%%%%% Header %%%%%%%%%%%%%%%%%%%%%%%%%%%%%%
\title{\EventName}
\subtitle{\TPindex: \TPtitle}

\author{\authors}
\institute[UL]{
  University of Luxembourg, Luxembourg
}

% Mandatory to define a logo - otherwise compilation will fail in an unobvious
% manner.
\pgfdeclareimage[height=0.8cm]{logo}{images/logo_UL.pdf}
\logo{\pgfuseimage{logo}}
\date{}

%%%%%%%%%%%%%%%%%%%%%%%%%%%%%% Body %%%%%%%%%%%%%%%%%%%%%%%%%%%%%%%
\begin{document}

\begin{frame}
    \vspace{2.5em}
    \titlepage
\end{frame}

% .......
\frame{
  \begin{center}
      \textbf{Latest versions available on
        \href{https://github.com/ULHPC/}{Github}}:
      \vfill
      \begin{description}
        \item[UL HPC tutorials:] \hfill
          \myurl{https://github.com/ULHPC/tutorials}
        \item[UL HPC School:] \hfill
          \myurl{http://hpc.uni.lu/hpc-school/}
        \item[\TPindex tutorial sources:] \hfill
          \myurl{\TPghurl}
      \end{description}
  \end{center}
}

% ......
\frame{
  \frametitle{Summary}
  {\scriptsize
    \tableofcontents
  }
}

% ===============================================
\section{Pre-requisites}

% Example of prompt usage
% .......
\begin{frame}[fragile]
    \frametitle{MPI tasks: 3 Suites via \texttt{module}}
    \begin{enumerate}
      \item OpenMPI\hfill\myurl{http://www.open-mpi.org/}\\
        \begin{cmdline}
            \cmdlinenode{module load OpenMPI}\\
            \cmdlinenode{make}\\
            \cmdlinenode{mpirun -machinefile \$OAR\_NODEFILE /path/to/mpi\_prog}

        \end{cmdline}

      \item MVAPICH2\hfill\myurl{http://mvapich.cse.ohio-state.edu/overview/mvapich2}\\
        \begin{cmdline}
            \cmdlinenode{module purge}\\
            \cmdlinenode{module load MVAPICH2}\\
            \cmdlinenode{make clean \&\& make}\\
            \cmdlinenode{mpirun -machinefile \$OAR\_NODEFILE /path/to/mpi\_prog}

        \end{cmdline}

      \item  Intel Cluster Toolkit Compiler Edition (\texttt{ictce} for short):
        \begin{cmdline}
            \cmdlinenode{module purge}\\
            \cmdlinenode{module load ictce}\\
            \cmdlinenode{make clean \&\& make}\\
            \cmdlinenode{mpirun -machinefile \$OAR\_NODEFILE /path/to/mpi\_prog}

        \end{cmdline}
    \end{enumerate}
\end{frame}

\section{Parallelism using MPI}
\subsection{Basic execution}
\begin{frame}
  \frametitle{Basic // execution with MPI}

  \begin{block}{MPI allows \emph{multi-node} parallel execution}
       \begin{itemize}
        \item \emph{MPI-enabled} applications are (generally) run through \emph{mpirun}
          \begin{cmdline}
              \cmdlineentry{mpirun \textit{[options]} ./application}
          \end{cmdline}
        \pause
        \item Common options are \texttt{-n} and \texttt{-hostfile}
        \begin{cmdline}
              \cmdlineentry{mpirun -n \textit{NUM} ./application}
              
	      \cmdlineentry{mpirun -hostfile \textit{FILE} ./application}
	      
	      \cmdlineentry{mpirun -n \textit{NUM} -hostfile \textit{FILENAME} ./application}
        \end{cmdline}
        \pause
        \item If only \texttt{-n} is specified, the // processes are run \emph{locally}
        \pause
        \item If only \texttt{-hostfile} is specified, as many // processes are started as defined in the hostfile (\emph{locally} and/or \emph{remotely})
        \pause
        \item Using both options, one explicitly defines how many // processes are run (\emph{locally} and/or \emph{remotely})
      \end{itemize}
  \vspace{-2ex}
  \end{block}

\end{frame}


\subsection{A concrete example}

\begin{frame}[fragile]
  \frametitle{A concrete example on UL HPC (I)}

  \begin{enumerate}
   \item Request computing resources and check them
   \begin{lstlisting}[basicstyle=\tiny,numbers=none]
   (access)$> oarsub -I -l nodes=2/core=2
   (gaia-12)$> cat $OAR_NODEFILE
   gaia-12
   gaia-12
   gaia-15
   gaia-15
\end{lstlisting}
 \pause
 \item Run the application without MPI, then load an MPI stack
   \begin{lstlisting}[basicstyle=\tiny,numbers=none]
   (gaia-12)$> ./mpi_hello_world 
   Hello world from processor gaia-12, rank 0 out of 1 processors
   (gaia-12)$> module load OpenMPI/1.7.3-GCC-4.8.2
   \end{lstlisting}
 \pause
 \item Run the application with MPI, locally
   \begin{lstlisting}[basicstyle=\tiny,numbers=none]
   (gaia-12)$> $ mpirun -n 3 ./mpi_hello_world 
   Hello world from processor gaia-12, rank 2 out of 3 processors
   Hello world from processor gaia-12, rank 1 out of 3 processors
   Hello world from processor gaia-12, rank 0 out of 3 processors
   \end{lstlisting}
 \end{enumerate}
\end{frame}

\begin{frame}[fragile]
  \frametitle{A concrete example on UL HPC (II)}
   \begin{enumerate}
     \setcounter{enumi}{3}
     \item Run the application with MPI on all available resources
      \begin{lstlisting}[basicstyle=\tiny,numbers=none]
        (gaia-12)$> mpirun -hostfile $OAR_NODEFILE ./mpi_hello_world 
	Hello world from processor gaia-12, rank 1 out of 4 processors
	Hello world from processor gaia-12, rank 0 out of 4 processors
	Hello world from processor gaia-15, rank 2 out of 4 processors
	Hello world from processor gaia-15, rank 3 out of 4 processors
        \end{lstlisting}
	\pause
      \item  Run the application with MPI on part of the available resources
        \begin{lstlisting}[basicstyle=\tiny,numbers=none]
	(gaia-12)$> mpirun -n 3 -hostfile $OAR_NODEFILE ./mpi_hello_world 
	Hello world from processor gaia-12, rank 0 out of 3 processors
	Hello world from processor gaia-12, rank 1 out of 3 processors
	Hello world from processor gaia-15, rank 2 out of 3 processors
	\end{lstlisting}
  \end{enumerate}
\end{frame}


\section{Parallelism using MPI/OpenMP}

\begin{frame}
  \frametitle{MPI and OpenMP hybrids}
  \vspace{-2ex}
  \begin{block}{Reminder}
    \begin{itemize}
      \item MPI enables \emph{single and multi-node} parallel execution
      \item OpenMP allows \emph{single-node} parallel execution
    \end{itemize}
  \vspace{-2ex}
  \end{block}
  \pause
   \begin{block}{Hybrid MPI/OpenMP software}
      \begin{itemize}
	\item Efficient use of resources by:
	  \begin{itemize}
	    \item using OpenMP to run things in parallel \emph{locally}
	    \item using 1 MPI process per compute node to enable \emph{multi-node} parallelism
	  \end{itemize}
      \pause
	\item Require additional specific instructions when launching the application
	  \begin{itemize}
	    \item e.g. use \emph{mpirun -npernode 1} for MPI
	    \item use environment variables (e.g. OMP\_NUM\_THREADS) or application-specific parameters for OpenMP
	  \end{itemize}
      \end{itemize}
  \vspace{-2ex}
  \end{block}
\end{frame}


\section*{Thank you for your attention...}
\frame{
  \frametitle{Questions?}
  \begin{center}
      \includegraphics[scale=0.2]{question.jpg}
  \end{center}

  {\tiny
    \tableofcontents

  }
}

\newcounter{finalframe}
\setcounter{finalframe}{\value{framenumber}}

% \appendix

\frame{
  \frametitle{Appendix}

  \begin{acronym}\setlength\itemsep{-0.3em}
      \acro{UL}{University of Luxembourg}
  \end{acronym}
}

% .......
\frame{
  \frametitle{Another appendix slide}

}


\setcounter{framenumber}{\value{finalframe}}

\end{document}

% ~~~~~~~~~~~~~~~~~~~~~~~~~~~~~~~~~~~~~~~~~~~~~~~~~~~~~~~~~~~~~~~~
% eof
% 
% Local Variables:
% mode: latex
% mode: flyspell
% mode: auto-fill
% fill-column: 80
% End:
