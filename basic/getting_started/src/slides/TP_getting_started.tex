% =============================================================================
% File:  sample_slides.tex --  Example of the use of the Falkor Beamer theme
% Author(s): Sebastien Varrette <Sebastien.Varrette@uni.lu>
% Time-stamp: <Mar 2014-04-29 16:06 svarrette>
% 
% Copyright (c) 2012 Sebastien Varrette <Sebastien.Varrette@uni.lu>
% .             http://varrette.gforge.uni.lu
% 
% For more information:
% - LaTeX: http://www.latex-project.org/
% - Beamer: https://bitbucket.org/rivanvx/beamer/
% - LaTeX symbol list:
% http://www.ctan.org/tex-archive/info/symbols/comprehensive/symbols-a4.pdf
% 
% Latest version of these files can be found on Github:
% 

% =============================================================================
\documentclass{beamer}
% \documentclass[draft]{beamer}
\usepackage{_style}
\usepackage{__config}

% The key part to use my theme
\usetheme{Falkor}

% Not integrated in my theme as not everybody wants that
\AtBeginSection[]
{
  \frame{
    \frametitle{Summary}
    {\scriptsize\tableofcontents[currentsection]}
  }
}

%%%%%%%%%%%%%%%%%%%%%%%%%%%% Header %%%%%%%%%%%%%%%%%%%%%%%%%%%%%%
\title{\EventName}
\subtitle{\TPindex: \TPtitle}

\author{\authors}
\institute[UL]{
  University of Luxembourg, Luxembourg
}

% Mandatory to define a logo - otherwise compilation will fail in an unobvious
% manner.
\pgfdeclareimage[height=0.8cm]{logo}{images/logo_UL.pdf}
\logo{\pgfuseimage{logo}}
\date{}

%%%%%%%%%%%%%%%%%%%%%%%%%%%%%% Body %%%%%%%%%%%%%%%%%%%%%%%%%%%%%%%
\begin{document}

\begin{frame}
    \vspace{2.5em}
    \titlepage
\end{frame}

% .......
\frame{
  \begin{center}
      \textbf{Latest versions available on
        \href{https://github.com/ULHPC/}{Github}}:
      \vfill
      \begin{description}
        \item[UL HPC tutorials:] \hfill
          \myurl{https://github.com/ULHPC/tutorials}
        \item[UL HPC School:] \hfill
          \myurl{http://hpc.uni.lu/hpc-school/}
        \item[\TPindex tutorial sources:] \hfill
          \myurl{\TPghurl}
      \end{description}
  \end{center}
}

% ......
\frame{
  \frametitle{Summary}
  {\scriptsize
    \tableofcontents
  }
}

% ===============================================
\section{Pre-requisites}

% Example of prompt usage
% .......
\begin{frame}[fragile]
    \frametitle{Pre-requisites}

    \begin{enumerate}
      \item HPC account
      \item Have a look at the website hpc.uni.lu
      \item Preferably use a Mac or Linux environment
    \end{enumerate}
\end{frame}





% ===============================================
\section{Objectives}

% ............
\frame{

  \frametitle{Objectives of the TP}

  \begin{itemize}
    \item Connecting to the \ULHPC
      \begin{itemize}
    		\itemhook SSH configuration
    		\itemhook Internal SSH key
    		\itemhook overcome port filtering with ProxyCommand
	    \end{itemize}
    \item Discovering, visualizing and reserving UL HPC resources
      \begin{itemize}
    		\itemhook Working environment
    		\itemhook Web monitoring interfaces
    		\itemhook using OAR
    		\itemhook Job management
    		\itemhook Modules
	    \end{itemize}

  \end{itemize}

}

\section{Connecting to the ULHPC}

% .......
\frame{
  \frametitle{SSH set-up}

      \begin{itemize}
    	\item SSH configuration

            \begin{cmdline}
              \cmdlineentry{ssh chaos-cluster}\\
              \cmdlineentry{ssh gaia-cluster}\\
          \end{cmdline}
    	
    	
    	\item Internal SSH key
    	
   		\item Overcome port filtering with ProxyCommand
      \end{itemize}
      
}

\section{Discovering, visualizing and reserving UL HPC resources}

% .......
\frame{
  \frametitle{Basic usage}

      \begin{itemize}
    		\item Working environment
    		\item Web monitoring interfaces
    		\item using OAR
    		\item Job management
    		\item Modules
      \end{itemize}
}


\section*{Thank you for your attention...}
\frame{
  \frametitle{Questions?}
  \begin{center}
      \includegraphics[scale=0.2]{question.jpg}
  \end{center}

  {\tiny
    \tableofcontents

  }
}

\end{document}
$\backslash$\\~~~~~~~/path/to/cloning/dir/beamerthemeFalkor/ .}\\
              \cmdlineentry{make}
          \end{cmdline}

}


% ~~~~~~~~~~~~~~~~~~~~~~~~~~~~~~~~~~~~~~~~~~~~~~~~~~~~~~~~~~~~~~~~
% eof
% 
% Local Variables:
% mode: latex
% mode: flyspell
% mode: auto-fill
% fill-column: 80
% End:
