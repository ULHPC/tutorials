% =============================================================================
% File:  sample_slides.tex --  Example of the use of the Falkor Beamer theme
% Author(s): Sebastien Varrette <Sebastien.Varrette@uni.lu>
% Time-stamp: <Mar 2014-04-29 15:41 svarrette>
% 
% Copyright (c) 2012 Sebastien Varrette <Sebastien.Varrette@uni.lu>
% .             http://varrette.gforge.uni.lu
% 
% For more information:
% - LaTeX: http://www.latex-project.org/
% - Beamer: https://bitbucket.org/rivanvx/beamer/
% - LaTeX symbol list:
% http://www.ctan.org/tex-archive/info/symbols/comprehensive/symbols-a4.pdf
% 
% Latest version of these files can be found on Github:
% 

% =============================================================================
\documentclass{beamer}
% \documentclass[draft]{beamer}
\usepackage{_style}
\usepackage{__config}

% The key part to use my theme
\usetheme{Falkor}

% Not integrated in my theme as not everybody wants that
\AtBeginSection[]
{
  \frame{
    \frametitle{Summary}
    {\scriptsize\tableofcontents[currentsection]}
  }
}

%%%%%%%%%%%%%%%%%%%%%%%%%%%% Header %%%%%%%%%%%%%%%%%%%%%%%%%%%%%%
\title{\EventName}
\subtitle{\TPindex: \TPtitle}

\author{\authors}
\institute[UL]{
  University of Luxembourg, Luxembourg
}

% Mandatory to define a logo - otherwise compilation will fail in an unobvious
% manner.
\pgfdeclareimage[height=0.8cm]{logo}{images/logo_UL.pdf}
\logo{\pgfuseimage{logo}}
\date{}

%%%%%%%%%%%%%%%%%%%%%%%%%%%%%% Body %%%%%%%%%%%%%%%%%%%%%%%%%%%%%%%
\begin{document}

\begin{frame}
    \vspace{2.5em}
    \titlepage
\end{frame}

% .......
\frame{
  \begin{center}
      \textbf{Latest versions available on
        \href{https://github.com/ULHPC/}{Github}}:
      \vfill
      \begin{description}
        \item[UL HPC tutorials:] \hfill
          \myurl{https://github.com/ULHPC/tutorials}
        \item[UL HPC School:] \hfill
          \myurl{http://hpc.uni.lu/hpc-school/}
        \item[\TPindex tutorial sources:] \hfill
          \myurl{\TPghurl}
      \end{description}
  \end{center}
}

% ......
\frame{
  \frametitle{Summary}
  {\scriptsize
    \tableofcontents
  }
}

% ===============================================
\section{Pre-requisites}

% Example of prompt usage
% .......
\begin{frame}[fragile]
  \frametitle{MPI tasks: 3 Suites via \texttt{module}}
      \begin{enumerate}
        \item OpenMPI\hfill\myurl{http://www.open-mpi.org/}\\
          \begin{cmdline}
              \cmdlinenode{module load OpenMPI}\\
              \cmdlinenode{make}\\
              \cmdlinenode{mpirun -machinefile \$OAR\_NODEFILE /path/to/mpi\_prog}
              
          \end{cmdline}

        \item MVAPICH2\hfill\myurl{http://mvapich.cse.ohio-state.edu/overview/mvapich2}\\
          \begin{cmdline}
              \cmdlinenode{module purge}\\
              \cmdlinenode{module load MVAPICH2}\\
              \cmdlinenode{make clean \&\& make}\\
              \cmdlinenode{mpirun -machinefile \$OAR\_NODEFILE /path/to/mpi\_prog}
              
          \end{cmdline}

        \item  Intel Cluster Toolkit Compiler Edition (\texttt{ictce} for short):
          \begin{cmdline}
              \cmdlinenode{module purge}\\
              \cmdlinenode{module load ictce}\\
              \cmdlinenode{make clean \&\& make}\\
              \cmdlinenode{mpirun -machinefile \$OAR\_NODEFILE /path/to/mpi\_prog}
              
          \end{cmdline}
     \end{enumerate}
\end{frame}


% .......
\frame{
  \frametitle{Full sample example \hfill{\tiny (\textit{i.e.} these slides)}}

  \begin{block}{}
      \begin{itemize}
        \item To copy a full working example
          \begin{cmdline}
              \cmdlineentry{cd /path/to/cloning/dir}\\
              \cmdlineentry{git clone https://github.com/Falkor/beamerthemeFalkor.git}\\
              \cmdlineentry{cd /path/to/working/dir}\\
              \cmdlineentry{rsync -avzu -L --exclude "*.git" $\backslash$\\~~~~~~~/path/to/cloning/dir/beamerthemeFalkor/ .}\\
              \cmdlineentry{make}
          \end{cmdline}
      \end{itemize}
  \end{block}
  \begin{itemize}
    \item This will generate the file \texttt{sample\_slides.pdf}.
      \begin{itemize}
          \itemhook adapt accordingly...
      \end{itemize}
  \end{itemize}

}


% ===============================================
\section{Some example slides for the TP}

% ............
\frame{

  \frametitle{Objectives of the TP}

  \begin{itemize}
    \item Better understand the usage of XXX on the \ULHPC
  \end{itemize}
  \vfill Add logo of the soft  
  \begin{alertblock}{}
      \begin{itemize}
        \item Key objective 1
          \begin{itemize}
              \itemhook at the heart of ...
          \end{itemize}
        \item Key objective 2
          \begin{itemize}
              \itemhook select benchmarking tools to reflect an HPC usage
          \end{itemize}
      \end{itemize}
  \end{alertblock}


  ~\vfill
  {\tiny
    \mycite{SBAC-PAD13} S. Varrette,  M. Guzek, V. Plugaru, J. E. Sanchez, and P. Bouvry.
    "\textit{HPC Performance and Energy-Efficiency of Xen, KVM and VMware Hypervisors}". In Proc. of the 25th IEEE Symposium on Computer Architecture and High Performance (SBAC-PAD'13), Oct 2013.
    % Do not forget the below space

  }

}

\frame[containsverbatim]{
  \frametitle{A slide with listings}%\frametitle{Problem}
  \begin{block}{A JavaScript program}
      \begin{lstlisting}[basicstyle=\tiny,numbers=none]
          function fibo(n)
          {
            if( n <= 1 )
            {
              return n;
            }
            var res = fibo(n-1) + fibo(n-2);
            return res;
          }
          n = parseFloat(arguments[1])
          nn = fibo(n)
          print(nn)
      \end{lstlisting}

  \end{block}

}

\section{Example 1}

%.......
\frame{
  \frametitle{Example 1}
  \vfill{\tiny 
    \href{https://github.com/ULHPC/tutorials/tree/devel/advanced/HPL#runs-on-a-single-node}{[Online]}}
  \begin{exampleblock}{Title}
      \begin{itemize}
        \item recall description of the task (following github link)
      \end{itemize}
  \end{exampleblock}
  \pause
  Eventually give some hint on the solutioon
}

\section{Example 2}

%.......
\frame{
  \frametitle{Example 2}

  \begin{exampleblock}{Title}
      \begin{itemize}
        \item recall description of the task
      \end{itemize}
  \end{exampleblock}
  \pause
  Eventually give some hint on the solutioon
}


\section{Conclusion: Final plot}

% ............
\begin{frame}
    \frametitle{Conclusion}

    \begin{itemize}
      \item Summary point 1
      \item Summary point 2
    \end{itemize}

    \begin{block}{Perspectives}
        \begin{itemize}
          \item Improve point 1
          \item Improve point 2
        \end{itemize}
    \end{block}


\end{frame}

\section*{Thank you for your attention...}
\frame{
  \frametitle{Questions?}
  \begin{center}
      \includegraphics[scale=0.2]{question.jpg}
  \end{center}

  {\tiny
    \tableofcontents

  }
}

\newcounter{finalframe}
\setcounter{finalframe}{\value{framenumber}}

% \appendix

\frame{
  \frametitle{Appendix}

  \begin{acronym}\setlength\itemsep{-0.3em}
      \acro{DFT}{Discrete Fourier Transform}
      \acro{EA}{Evolutionary Algorithm}
      \acro{PRNG}{[Pseudo]-Random Number Generator}
      \acro{UL}{University of Luxembourg}
  \end{acronym}

  \textit{*Note: notice the slide number below...}
}

% .......
\frame{
  \frametitle{Another appendix slide}

  \textit{Note again the slide number below...}

}


\setcounter{framenumber}{\value{finalframe}}

\end{document}

% ~~~~~~~~~~~~~~~~~~~~~~~~~~~~~~~~~~~~~~~~~~~~~~~~~~~~~~~~~~~~~~~~
% eof
% 
% Local Variables:
% mode: latex
% mode: flyspell
% mode: auto-fill
% fill-column: 80
% End:
